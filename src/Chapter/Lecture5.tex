\section{Aithen's Methods}\label{sec:aithen's-methods}
If one already uses an algorithm that shows linear convergence, Aithen's method might be useful to
accelerate convergence.
Linear convergence:
\begin{equation*}
    \lvert x_{i+1} - \hat{x} \rvert \leq C * \lvert x_i - \hat{x} \rvert
\end{equation*}
Assuming:
\begin{align*}
    \frac{ x_{i+1}-\hat{x} }{x_i - \hat{x}} &\approx
    \frac{ x_{i+2}-\hat{x}}{x_{i+1}-\hat{x}}\\
    (x_{i+1}-\hat{x})^2 &\approx (x_{i+2}-\hat{x})(x_i-\hat{x})\\
    x_{i+1}^2-2x_{i+1}\hat{x} &\approx x_i x_{i+2}-(x_i+x_{i+2})\hat{x}\\
    \hat{x}(x_i-2x_{i+1}+x_{i+2}) &\approx x_i x_{i+2}- x_{i+1}^2\\
    \hat{x} &\approx \frac{x_i x_{i+2}-x_{i+1}^2}{x_i-2x_{i+1}+ x_{i+2}}\\
    \hat{x}\approx\frac{x_i x_{i+2}-2x_i x_{i+1}+x_i^2-x_{i+1}^2+2x_i x_{i+2}-x_i^2}{x_i-2x_{i+1}+x_{i+2}}&= x_i-\frac{(x_{i+1}-x_i)^2}{x_{i+2}-2x_{i+1}+x_i}\\
\end{align*}
This method is also called $\Delta^2$-method.
\begin{align*}
    \Delta y_n &= y_{n+1}-y_n\\
    \Delta^2 y_n &= \Delta(\Delta y_n) = \Delta y_{n+1}-\Delta y_n = (y_{n+2}-y_{n+1})-(y_{n+1}-y_n)\\
    &=y_{n+2} -2y_{n+1}+y_n
\end{align*}
\begin{align*}
    \to \hat{x}\approx x_i - \frac{(\Delta x_i)^2}{\Delta^2 x_i}
\end{align*}


\chapter{Interpolation}\label{ch:interpolation}
Often instead of a function f only individual values $f(x_i)$ and
perhaps values of the derivatives $f'(x_i), f''(x_i)$ are available.
This is for instance the case if differential equations are being solved
numerically or with experimental data.
However, for experimental data, one typically uses fitting and not interpolation.
If one wants to obtain function values at intermediate position, integrate or
simply get a smooth representation, one has to interpolate.
That means one searches for a function $\varphi$ that agrees with $f$, $f'(x), \ldots$
at the nodes $x_i$:
\begin{align*}
    f(x_i)=\varphi(x_i)\\
    f'(x_i)=\varphi'(x_i)\\
    \ldots
\end{align*}


\section{Polynomial interpolation}\label{sec:polynomial-interpolation}
Simple problem (no derivatives): $y_i = f(x_i)$ with $i = 0, \ldots, n$ given,
looking for polynomial $P \in \mathbb{P}_n$ of degree $\leq n$ such that
$P(x_i) = y_i$

$P$ is unambiguous.
To show this, we assume the negation:
\begin{equation*}
    P, Q \in \mathbb{P}_n \text{ such that } P(x_i)=Q(x_i)=y_i\text{, for all } i = 0, \ldots, n
\end{equation*}
Then $P-Q$ is a polynomial of degree $\leq n$ with roots at
$x_0, x_1, \ldots x_n$.
But non-zero polynomials of degree $n$ have at most
exactly $n$ roots.
Hence, $P(x)-Q(x) = 0$

There are different methods for finding $P$, which can be associated with different
bases of the space of polynomials.

\subsection{Vandermonde-Matrix}\label{subsec:vandermond-matrix}
Using the monomial basis $\{ 1, x, x^2, x^3, \ldots, x^n\}$ $P(x_i) = y_i$ forms a
linear system of equations:
\begin{equation*}
    \begin{pmatrix*}
        1 &x_0 &x_0^2 &\ldots & x_0^n\\
        1 &x_1 &x_1^2 &\ldots & x_1^n\\
        \vdots \\
        1 & x_n & x_n^2&\ldots  &x_n^2
    \end{pmatrix*}
    *
    \begin{pmatrix*}
        a_0 \\
        \vdots\\
        a_n
    \end{pmatrix*}
    =
    \begin{pmatrix*}
        y_0\\
        \vdots\\
        y_n
    \end{pmatrix*}
\end{equation*}
The solution (inversion of $U$) is computationally expensive ($\propto n^3$) operations).

\subsection{Lagrange polynomials}\label{subsec:lagrange-polynomials}
An alternative basis is given by the Lagrange-polynomials $L_0, L_1, \ldots, L_n$ which implicitly are defined through $L_i(x_j) = \delta_{ij}$.\\
The explicit form is:
\begin{equation*}
    \prod_{j=0, j\neq i}^n \frac{x-x_j}{x_i-x_j}
\end{equation*}
The interpolating polynomial $P$ is obtained by superposition
\begin{align*}
    P(x) &= \sum_{i=0}^{n}y_i*L_i(x)\\
    \to P(x_j)&=\sum_{i=0}^{n} y_i L_i(x_j)= \sum_{i=0}^{n} y_i * \delta_{ij} = y_j\\
    \\
    P(x) &= y_0 L_0 (x) + y_1 L_1 (x) + y_2 L_2 (x) + \ldots\\
    P(x_2) &= y_0 \underbrace{L_0
        (x_2)}_0 + y_1 \underbrace{L_1 (x_2)}_0 + y_2  \underbrace{L_2 (x_2)}_{1} + y_3  \underbrace{L_3 (x_2)}_0, \ldots\\
\end{align*}
\begin{remark}
    The Lagrange polynomials form an orthonormal basis with respect
    to the inner product $\langle P, Q \rangle =\sum_{i=0}^{n} P(x_i) * Q(x_i)$
    \begin{align*}
        \to{} \langle P, L_i \rangle &= P(x_i)\\
        P = \sum_{i=0}^n \langle P, L_{i} \rangle L_i &= \sum_{i=0}^{n} P(x_i) L_i
        = \sum_{i=0}^{n} y_i L_i(x)
    \end{align*}
\end{remark}

\begin{example}
    \begin{equation*}
        f(x)= \frac{1}{x}
    \end{equation*}
    \begin{center}
        \begin{tabular}{l l l}
            \toprule
            $i$ & $x_i$ & $y_i$          \\
            \midrule
            0   & 2     & $\sfrac{1}{2}$ \\
            1   & 2.5   & $\sfrac{2}{5}$ \\
            2   & 4     & $\sfrac{1}{4}$ \\
            \bottomrule
        \end{tabular}
    \end{center}

    \begin{align*}
        L_0(x) &= \frac{(x-2.5)(x-4)}{(2-2.5)(2-4)}=(x-6.5)*x+10\\
        L_1(x) &= \frac{(x-2)(x-4)}{(2.5-2)(2.5-4)}=\frac{(-4x+24)x-32}{3}\\
        L_2(x) &= \frac{(x-2)(x-2.5)}{(4-2)(4-2.5)}= \frac{(x-4.5)x+5}{3}\\
        P=\sum_{i=0}^{2}y_i*L_i(x) &= \frac{1}{2}((x-6.5)x+10)+ \frac{2}{5}
        \left( \frac{(-4x+24)x-32}{3} \right)+\frac{1}{4}\left( \frac{(x-4.5)x+5}{3} \right)\\
        &=(0.05x-0.425)x+1.15
    \end{align*}
\end{example}

\subsection{Neville's method}\label{subsec:neville's-method}
Let $P_{m_0, m_1, \dots, m_k} (x)$ be the polynomial of degree $k$ that interpolates $f(x)$\\
at the nodes $x_{m_0}, x_{m_1},\ldots,x_{m_k}$
\begin{equation*}
    P(x)= \frac{(x_j-x)P_{0,1,\ldots, j-1, j+1, \ldots, n}(x)-(x_i-x)P_{0,1,\ldots, i-1, i+1, \ldots, n}(x)}{(x_j-x_i)}
\end{equation*}
is the polynomial of degree n that interpolates f at the nodes $x_0, \ldots, x_n$ (proof by plugging in).

Involves a lot of operations but is efficient if one is only interested in the value of the interpolating polynomial at one specific position $\hat{x}$.
Then $P_{1,2}(\hat{x})$, $P_{0,1,\ldots, n-1}(\hat{x})$, \ldots{} are just numbers.

\subsection{Newton's method}\label{subsec:newton's-method}
Newton basis: $N_0 (x) = 1$, $N_1(x) =(x-x_0)$, $N_2(x)=(x-x_0)(x-x_1)$, \ldots
\begin{align*}
    N_i(x)=\prod_{k=0}^{i-1}(x-x_k)\\
    \to P_{0,1,\ldots, i-1}(x)+c_i * N_i(x) = P_{0,1,\ldots, i}(x)\\
    \text{since } N_i(x_0)=N_i(x_1)=\ldots=N_i(x_{i-1})=0
\end{align*}
The problem becomes what is $c_i$?
effective determination through divided differences($f[\ldots]$). Polynomial of degree 0:
\begin{align*}
    P_i(x) &= f[x_i] \coloneqq f_i = f(x_i)\\
    P_{i, i+1} &= \frac{(x_i - x) P_{i+1}-(x_{i+1} P_i(x))}{x_i-x_{i+1}} =f[x_i]+(x-x_i)\underbrace{f[x_i, x_{i+1}]}_{=c}\\
    &=\frac{(x_i-x)f[x_{i+1}]-(x_{i+1}-x) f[x_i]}{x_i-x_{i+1}}\\
    f[x_i, x_{i+1}] &= \frac{(x_i-x)f[x_{i+1}]-(x_{i+1}-x) f[x_i]-(x_i-x_{i+1}) f[x_i]}{(x_i-x_{i+1})(x-x_i)}\\
    &= \frac{(x_i-x)f[x_{i+1}]-(x_i-x)f[x_i]}{(x_i-x_{i+1})(x-x_i)}=\frac{f[x_i]-f[x_{i+1}]}{x_i-x_{i+1}}
\end{align*}
Let the ($n-1$)th divided difference $f[x_0, \ldots, x_{n-1}]$ and $f[x_i, \ldots, x_n]$ be known.
\begin{align*}
    P(x)=P_{0, \ldots, n}(x)&=P_{0,1,\ldots, n-1}(x)+f[x_0, \ldots, x_n](x-x_0)(x-x_1)\ldots(x-x_{n-1}) \text{(\textasteriskcentered{})}\\
    P_{0,\ldots, n-1}(x)&=f[x_0]+f[x_0, x_1](x-x_0)+f[x_0,x_1,x_2](x-x_0)(x-x_1)+
    \ldots\\
    &\phantom{=}+f[x_0,\ldots, ]x_{n-1}(x-x_0)(x-x_1)\ldots(x-x_{n-2})\\
    P_{1,\ldots, n} &= f[x_1] + f[x_1, x_2](x-x_1)+f[x_1, x_2, x_3](x-x_1)(x-x_2)+\ldots\\
    &\phantom{=}+f[x_1, x_2, \ldots, x_n] (x-x_1)(x-x_2)\ldots (x-x_{n-1})\\
    P_{0,1,\ldots, n}(x) &= \frac{(x_0-x)P_{1,\ldots, n}(x)-(x_n-x)P_{0,\ldots, n-1}}{x_0-x_n} \text{(\textasteriskcentered{}\textasteriskcentered{})}\\
    &\text{compare coefficients of $x^n$ in (\textasteriskcentered{}) and (\textasteriskcentered{}\textasteriskcentered{})}\\
    f[x_0,\ldots, x_n]&=\frac{f[x_0, \ldots, x_{n-1}]-f[x_1, \ldots, x_n]}{x_0-x_1}
\end{align*}
