\chapter{Stability and fixed-point theorem}\label{ch:stability-and-fixed-point-theorem}


\section{Stability}\label{sec:stability}
Consider
\begin{equation*}
    \int_0^1 \frac{x^{10} }{x+10} dx
\end{equation*}
This integral can be solved iteratively:
\begin{align*}
    y_n &= \int_0^1 \frac{x^n}{x+10} dx \\
    y_0 &= \int_0^1 \frac{1}{x+10} dx = \left[ \ln(x+10) \right]^1_0 = \ln\left( \frac{11}{10} \right) \approx 0.095310\\
    y_n + 10 * y_{n-1} &= \int_0^1 \frac{x^n+10*x^{n-1}}{x+10} dx = \int_0^1 x^{n-1} dx = \frac{1}{n}
\end{align*}
No we could use $y_n = \frac{1}{n}- 10 * y_{n-1}$ to obtain $y_1, \ldots{},  y_{10}$:
\begin{center}
    \begin{tabular}{r l}
        \toprule
        n  & $y_n$           \\
        \midrule
        0  & 0.0953102       \\
        1  & 0.0468982       \\
        2  & 0.0310181       \\
        3  & 0.0231520       \\
        4  & 0.0184804       \\
        5  & 0.0151955       \\
        6  & 0.0147117       \\
        7  & -0.00425944     \\
        8  & 0.167594        \\
        9  & -1.56483 \ldots \\
        10 & 15\ldots        \\
        11 & -157\ldots      \\
        \bottomrule
    \end{tabular}
\end{center}
If we use $y_n = \frac{1}{n}- 10 * y_{n-1}$, the initial error of $y_0$ grows like $10^n$ and since $y_n< y_{n+1}$, the error soon dominates.
This is unstable behaviour.

If we use
\begin{equation*}
    y_n = \frac{1}{10}*\left( \frac{1}{n+1}-y_{n+1} \right)
\end{equation*}
and start with the crude approximation $y_{30}=0$, the large error of $y_{30}$ is reduced by a factor of 10 with each iteration.
This is a stable algorithm.
\begin{center}
    \begin{tabular}{r l}
        \toprule
        n  & $y_n$      \\
        \midrule
        30 & 0          \\
        29 & 0.00333    \\
        28 & 0.00311    \\
        27 & 0.00324    \\
        \ldots \\
        20 & 0.00434704 \\
        \ldots \\
        0  & 0.0953102  \\
        \bottomrule
    \end{tabular}
\end{center}

A numerical algorithm is called stable, if the provided solution to a problem P is the exact
solution to a Problem Q that can be derived from P by a slight variation of the input data.
Else it is called unstable.


\section{Fixed-point iteration and fixed-point theorem}\label{sec:fixed-point-iteration-and-fixed-point-theorem}
We search the root of $f(x) = 2x-\tan(x)$. $2x-\tan(x) = 0$ can be transformed into a fixpoint equation:
\begin{align*}
    &x = \Phi_1(x) = \frac{1}{2} \tan(x) \\
    &x = \Phi_2(x) = \arctan(2x)
\end{align*}
With $\hat{x} = \Phi(\hat{x})$
The idea of the fixed-point iteration is to create a sequence by repeated insertion in $\Phi$
\begin{equation*}
    x_{i+1} = \Phi(x_i)
\end{equation*}
which converges against $\hat{x}$

\begin{center}
    \begin{tabular}{r l l }
        \toprule
        $i$ & $x_{i+1}=\frac{1}{2}\tan(x)$ & $x_{i+1}=\arctan(2x_i)$ \\
        \midrule
        0   & 1.2                          & 1.2                     \\
        1   & 1.2861                       & 1.1760                  \\
        2   & 1.7084                       & 1.1688                  \\
        3   & -3.6108                      & 1.1666                  \\
        4   & {}!                          & 1.1659                  \\
        5   &                              & 1.1657                  \\
        6   &                              & 1.1656                  \\
        \bottomrule
    \end{tabular}
\end{center}
Consider the difference
\begin{equation*}
    |x_{i+1}-x_i| = |\Phi(x_i)-\Phi(x_{i-1})|
\end{equation*}

\begin{definition}
    Let $I=[a,b] \subset \mathbb{R}$ be an interval.
    $\Phi: I \to \mathbb{R}$ is a contraction on I if there is a $0 \leq \theta < 1$
    such that $|\Phi(x)-\Phi(y)| \leq \theta |x-y|$.
\end{definition}

\begin{remark}
    $\theta$ is called Lipschitz constant $\to$ Lipschitz continuity if $0 \leq \theta < \infty$
\end{remark}

\begin{lemma}
    If $\Phi: I \to  \mathbb{R}$ is continuously differentiable on I ($\Phi \in C^1(I)$) then
    \begin{equation*}
        \sup_{\substack{x,y \in I}} \frac{|\Phi(x)-\Phi(y)|}{|x-y|} = \sup_{\substack{z \in I}} \Phi'(z)
    \end{equation*}
\end{lemma}
\begin{proof}
    mean value theorem: For all $x,y \in I$, $x<y$ there is a $\xi \in I$ such that
    \begin{equation*}
        \Phi(x)-\Phi(y) = \Phi'(\xi)(x-y
    \end{equation*}
\end{proof}
\begin{theorem}[Fixed-point theorem]
    Let $I=[a,b] \subset \mathbb{R}$ be an interval and $\Phi: I \to I$ a contraction with Lipschitz constant $0 < \theta < 1$
    Then:
    \begin{enumerate}[label=(\roman*)]
        \item there is exactly one fixed-point $\hat{x}$ of $\Phi$, such that $\Phi(\hat{x}) = \hat{x}$
        \item For any initial value $x_0 \in I$ the fixed-point iteration $x_{i+1} = \Phi(x_i)$ converges against $\hat{x}$ with
        \begin{align*}
            |x_{i+1}-x_i| &\leq \theta *|x_i-x_{i-1}|\hbox{ and}\\
            |\hat{x}-x_i| &\leq \frac{\theta^i}{1- \theta} * |x_1-x_0|
        \end{align*}
    \end{enumerate}
\end{theorem}
\begin{proof}
    For all $x_0 \in I$
    \begin{align*}
        |x_{i+1}-x_i|=|\Phi(x_{i})-\Phi(x_{i-1})| &\leq  \theta * |x_i-x_{i-1}| \text{ therefore}\\
        |x_{i+1}-x_i| &\leq \theta^i*|x_1-x_0|
    \end{align*}
    Now we show that $x_i$ is a Cauchy sequence:
    \begin{align*}
        |x_{i+k}-x_i| &\leq |x_{i+k}-x_{i+k-1}|+|x_{i+k-1}-x_{i+k-2}|+\ldots + |x_{i+1}-x_i|\\
        &\leq (\theta^{i+k-1}+\theta^{i+k-2}+\ldots+\theta^i)* |x_1-x_0|\\
        &= \theta^i*(\theta^{k-1}+\theta^{k-2}+\ldots+\theta^{0})*|x_1-x_0| \\
        &\leq \frac{\theta^i}{1-\theta}* |x_1-x_0|
    \end{align*}
    Therefore, $x_i$ is a Cauchy sequence which implies convergence. $\hat{x} = \lim_{\substack{i \to \infty}}x_i$ is also a fixpoint of $\Phi$, since:
    \begin{align*}
        |\hat{x} - \Phi(\hat{x})|&=|\hat{x}-x_{i+1}+x_{i+1}-\Phi(\hat{x})|\\
        & = |\hat{x}-x_{i+1}+\Phi(x_i)-\Phi(\hat{x})|\\
        & \leq |\hat{x}-x_{i+1}| + |\Phi(x_i)-\Phi(\hat{x})|\\
        &  \leq |\hat{x}-x_i|+ \theta * |\hat{x}-x_i|\\
        &\to 0  \hbox{ for } i \to \infty
    \end{align*}
    Show that there is only one fixed-point by assuming the opposite.
    If $\hat{x}$ and $\hat{y}$ are distinct fixed-points, then
    \begin{equation*}
        0 \leq |\hat{x}-\hat{y}| = |\Phi(\hat{x})-\Phi(\hat{y})| \leq \theta * |\hat{x}-\hat{y}|
    \end{equation*}
    but $\theta < 1$, therefore this is only possible if $|\hat{x}- \hat{y}| = 0 \implies \hat{x} = \hat{y}$
\end{proof}


\section{Rate of convergence}\label{sec:rate-of-convergence}
A sequence $x_i \in \mathbb{R}$ converges with order (at least) $P \leq 1$ against $\hat{x}$ if there is a constant $C>0$ such that
\begin{equation*}
    |x_{i+1}-\hat{x}| \leq C * |x_i-\hat{x}|^P
\end{equation*}
where if $P=1$ it is additionally required that $C>1$.
If $P=1$ we speak of linear convergence, if $P=2$ of quadratic convergence.

Alternatively the rate of convergence can also be defined through the inequalities of differences of adjacent elements of the sequence.
\begin{equation*}
    |x_{i+1}-x_i| \leq C * |x_i-x_{i-1}|^P
\end{equation*}
In general, the fixed-point iteration converges only linearly due to
\begin{equation*}
    |x_{i+1}-\hat{x}| \leq \theta * |x_i-\hat{x}|
\end{equation*}


\section{Newtons method (for root finding)}\label{sec:newtons-method-(for-root-finding)}
We are searching the root $\hat{x}$ of f, but now we also have access to the first order derivative $f'(x)$.
First order taylor expansion at current position x:
\begin{align*}
    f(x) &\approx f(x_i)+ f'(x_i)(x-x_i)\\
    f(\hat{x}) &= 0\\
    x_{i+1} &= x_i - \frac{f(x_i)}{f'(x_i)}
\end{align*}
This is a fixed-point equation with the iteration function $\Phi(x)= x - \dfrac{f(x)}{f'(x)}$

\vspace{10mm}

\emph{Example:} Calculation of the squareroot of C\@.
\begin{equation*}
    f(x)=x^2-C
\end{equation*}
In floating point calculation the exponent of C should be treated separately:
\begin{equation*}
    \sqrt {C} = \sqrt {m * 2^e} = \sqrt {m} * 2^{\frac{e}{2}}
\end{equation*}
and $\sqrt{2}$ (for odd e) be calculated once and stored
\begin{align*}
    \sqrt {m} &= ? \\
    f(x) = x^2-m &= 0 \\
    f'(x) &= 2x \neq 0
\end{align*}
Where $m\in[1,2)$ and $x \in \left[1, \sqrt{2}\right]$

\begin{align*}
    \Phi(x)&=x- \frac{f(x)}{f'(x)}= x- \frac{x^2-m}{2x}= \frac{x}{2}+ \frac{m}{2x} = \frac{1}{2} \left( x+ \frac{m}{x} \right)\\
    x_{i+1} &= \frac{1}{2}*\left(x_i+\frac{m}{x_i} \right)
\end{align*}

E.g: $m=1.96$
\begin{center}
    \begin{tabular}{r l l}
        \toprule
        i & $x_i$              & \# correct digits \\
        \midrule
        0 & 1                & $\leq 1$          \\
        1 & 1.48             & $=1$              \\
        2 & 1.420216         & $=3$              \\
        3 & 1.40000167       & $=6$              \\
        4 & 1.40000000000099 & $=12$             \\
        5 & 1.4              & $=15$             \\
        \bottomrule
    \end{tabular}
\end{center}
(number of correct digits approximately doubles every iteration $\to$ quadratic convervenge)
