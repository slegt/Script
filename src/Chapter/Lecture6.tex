For $n+1$ data points $(x_i, f(x_i))$, $i = 0,1,\ldots, n$ with pairwise distinct $x_i$ there is exactly one interpolating polynomial $P(x)$
of degree $\leq n$ which is given by
\begin{align*}
    P(x) &= f[x_0]+f[x_0, x_1] (x-x_0) + f[x_0, x_1, x_2] (x-x_0)(x-x_1)+\\
    &\phantom{=}\ldots + f[x_0, x_1, \ldots, x_n] (x-x_0)(x-x_1)\ldots (x-x_{n-1})\\
\end{align*}
where the divided differences are given by
\begin{align*}
    f[x_i] = f(x_i)\\
    f[x_i, \ldots, x_{i+k}] = \frac{f[x_i, \ldots, x_{i+k-1}]-f[x_{i+1}, \ldots, x_{i+k}]}{x_i - x_{i+k}}
\end{align*}
Now we calculate the interpolating polynomials in the newton base, for previous example


\begin{center}
    \begin{tabular}{ccc}
        \toprule
        i & $x_i$ & $y_i$ \\
        \midrule
        1 & 2     & 1/2   \\
        2 & 2.5   & 2/5   \\
        3 & 4     & 1/4   \\
        \bottomrule
    \end{tabular}
\end{center}


\begin{center}
    \begin{tikzpicture}[
        node distance = 0pt,
        every node/.style = {draw, minimum size=15mm, minimum width=40mm, inner sep=5pt, outer sep=0pt, anchor=center}
    ]
        \node (n1)  {$f[x_0, x_1, x_2] = \frac{-0.1-(-0.2)}{4-2}=0.05$};
%
        \node (n11) [below left=of n1.south]    {$f[x_0, x_2] = \frac{0.4-0.5}{2.5-2}=-0.2$};
        \node (n12) [right=of n11]              {$f[x_1, x_2] = \frac{0.25-0.4}{4-2.5}=-0.1$};
%
        \node (n21) [below left=of n11.south]   {$f[x_0]=f(x_0) = 0.5$};
        \node (n22) [right=of n21]              {$f[x_1]= f(x_1) = 0.4$};
        \node (n23) [right=of n22]              {$f[x_2] = f(x_2) = 0.25$};
    \end{tikzpicture}
\end{center}
\begin{equation*}
    P(x)= 0.5-0.2(x-2)+0.05(x-2)(x-2.5)
\end{equation*}

\section{Error of polynomial interpolation}\label{sec:error-of-polynomial-interpolation}

\begin{theorem}
    Let $f:[a,b] \to \mathbb{R}$ be at least $(n+1)$ times continuously differentiable and $P(x)$ be the interpolating polynomial of f in the nodes
    $x_0, \ldots, x_n \in [a,b]$ of degree $\leq n$.
    Then for every $x \in [a,b]$ there is a $\xi = \xi(x) \in [a,b]$ such that
    \begin{equation*}
        f(x)-P(x) = \underbrace{(x-x_0)(x-x_1)\ldots(x-x_n)}_{\coloneqq W_{n+1}(x)}\frac{f^{n+1}(\xi)}{(n+1)!}
    \end{equation*}
\end{theorem}
\begin{proof}
    Let $F(x) \coloneqq f(x)-p(x)-k*W_{n+1}(x)$ and select some $\hat{x}\in [a,b], \hat{x} \neq x_i$.
    We choose K such that $F(\hat{x})=0$.
    This is always possible since $W_{n+1}(x) \neq 0$. $F(x)$ has therefore $n+2$ roots.
    This means that $F'(x)$ has $n+1$ roots, $F''(x)$ has n roots etc., until $F^{n+1}(x)= f^{n+1}(x)-k(n+1)!$ has one root,
    which we call $\xi = \xi(\hat{x})$
    \begin{align*}
        0 &= f^{n+1}(\xi)-k*(n+1)! \\
        k &= \frac{f^{n+1}(\xi)}{(n+1)!}
    \end{align*}
\end{proof}
\begin{remark}
    In order to estimate the error, one needs to know the boundaries of the $(n+1)$th derivative on $[a,b]$
\end{remark}


\section{Chebyshev/Tschebyscheff-polynomials}\label{sec:chebyshev/tschebyscheff-polynomials}
If we have the freedom to choose the nodes $x_i$, then this can be used to minimize $\lvert W_{n+1}(x) \rvert$ and thus
the error of the interpolating polynomial.
We constrict ourselves to $[a,b] = [-1,1]$.
If  $[a,b] \neq [-1,1]$ then the transformation
\begin{align*}
[-1,1]
    &\to [a,b]\\
    x &\to \frac{a+b}{2}+x \frac{b-a}{2}=y
\end{align*}
with the inverse transformation
\begin{align*}
[a,b]
    &\to [-1,1]\\
    y &\to \frac{2y}{b-a}- \frac{a+b}{b-a}=x
\end{align*}
does not change the properties of the interpolation and approximation.
The Chebyshev polynomials are recursively defined by
\begin{align*}
    T_0(x) &= 1\\
    T_1(x) &= x\\
    T_{n+1} &= 2x*T_n(x)-T_{n-1}(x)
\end{align*}
This implies $T_n (x)= \cos(n*\arccos(x))$
\begin{proof}
    \begin{align*}
        T_{n+1} &= 2x*\cos(n*\arccos(x))-\cos((n-1) \arccos(x))\\
        &= 2\cos (\arccos(x))*\cos(n*\arccos(x))-\cos((n-1))\underbrace{\arccos((x)}_\varphi)\\
        &= \cos((n+1) \varphi)+\cos((n-1)\varphi)-\cos((n-1)\varphi) = \cos((n+1) \varphi)
    \end{align*}
\end{proof}

Properties of $T_n$:
\begin{itemize}
    \item the roots of $T_n$ are $\cos(\frac{2k+1}{2k}*\pi)$ $(k=0,\ldots, n-1)$
    \item $T_n\left(\cos(\frac{k*\pi}{n})\right)= (-1)^k$
    \item $\lvert T_n(x) \rvert \leq 1$ for $x \in [-1,1]$
\end{itemize}
\begin{align*}
    T_0(x) &= 1\\
    T_1(x) &= x\\
    T_2(x) &= 2x^2-1\\
    T_3(x) &= 4x^{3}-3x\\
    T_4(x) &= 8x^4-8x^2+1
\end{align*}
\begin{theorem}
    From all possible $(x_0, \ldots, x_n)^T \in [-1,1]^n \subset \mathbb{R}^n$ the value $\max_{x \in [-1,1]} \lvert w_{n+1}(x) \rvert$ becomes minimal if the nodes $x_i$ are
    the roots of the $(n+1)$st Chebyshev polynomial $T_{n+1}$
    \begin{align*}
        x_i &= \cos\left(\frac{2i+1}{2n+2} \pi\right), i=0,\ldots,n\\
        \text{Then } w_{n+1}(x) &= \frac{1}{2^n}* T_{n+1}(x) \\
        \text{and } \max_{x \in [-1,1]} \lvert w_{n+1}(x) \rvert &= \frac{1}{2^n}
    \end{align*}
\end{theorem}

\begin{proof}
    \begin{lemma}
        Let $q(x)=2^{n-1}x^n+\ldots$  be a polynomial unequal to the nth Chebyshev polynomial $T_n$.
        Then:
        \begin{equation*}
            \max_{x \in [-1,1]} \lvert q(x) \rvert > 1= \max_{x \in [-1,1]} \lvert T_n(x) \rvert
        \end{equation*}
        We assume $\lvert q(x) \rvert \leq 1$ for all $x \in [-1,1]$, $T_n(1) = 1$, $T_n\left( \cos \frac{\pi}{n} \right) = -1$.
        We consider $T_n(x)-q_n(x)$ at $[\cos \frac{\pi}{n}, 1]$.
        $q(x)$ and $T_n(x)$ have the same highest order coefficient $(2^{n-1})$.
        Therefore $T_n(x)-q(x)$ is of degree $n-1$
        \begin{align*}
            &\lvert q(x) \rvert \leq 1\\
            &\to T_n(1)-q(1) \geq 0\\
            &\to T_n\left(\cos \left( \frac{\pi}{n} \right) \right) - q\left(\cos \left( \frac{\pi}{n} \right) \right) \leq 0
        \end{align*}
        $T_n(x)-q_(x)$ has at least one root on $[\cos\frac{\pi}{n}, 1]$.
        In the same way, we can show that $T_n(x)-q(x)$ has at least one root on $[\cos\frac{2\pi}{n},\cos\frac{\pi}{n}]$
        and on $[\cos\frac{3\pi}{n},\cos\frac{2\pi}{n}]$ and on \ldots{} and on $[-1, \cos \left( \frac{(n-1)  \pi}{n} \right)]$.
        Therefore $T_n(x)-q(x)$ has n roots in $[-1,1]$.
        If the root is situated on the boundary of two sub intervals, then it's a double root, since $T_n(x)$ and $q(x)$ extremal.
        However $T_n(x) - q(x)$ is only of degree $\leq n-1$
        \begin{align*}
            T_n(x)-q(x) = 0 \text{ (!) contradicition to assumption } T_n(x) \neq q(x) \\
            \to \max_{x \in [-1,1]} \lvert w_{n+1}(x) \rvert = \frac{1}{2^n}
            \underbrace{\max_{x \in [-1,1]} \lvert \underbrace{2^n w_{n+1}(x)}_{T_{n+1}(x)} \rvert}_{ \geq 1 \text{ with Lemma}}\\
            \max_{x \in [-1,1]} \lvert w_{n+1}  \rvert \geq \frac{1}{2^n} \text{ with equality if } w_{n+1}(x) = \frac{1}{2^n}*T_{n+1}(x)
        \end{align*}
    \end{lemma}
\end{proof}


\section{hermite interpolation}\label{sec:hermite-interpolation}
If in addition to the values of $f$ also the values of the derivative $f'$ are known, we can construct the hermite interpolating polynomial.
Every node $x_i \in \{ x_0, \ldots, x_n \}$ corresponds to $2$ conditions which implies that the interpolating polynom is of degree $2n+1$

\begin{theorem}
    Let $f \in C^1([a,b])$ and $x_0, \ldots, x_n \in [a,b]$ pairwise distinct.
    Then the only polynomial of degree $2n+1$ that equals $f$ and $f'$ in $x_0, \dlots, x_n$ is given by
    \begin{align*}
        \HO_{2n+1}(x) &= \sum_{j=0}^{n} f(x_j) * \HO_{n,j}(x) + \sum_{j=0}^{n} f'(x_j) * \widehat{\HO}_{n,j}(x)\\
        \text{where } \HO_{n,j}(x) &= [1-2(x-x_j) * L'_{n,j}(x_j)]*L_{n,j}(x)^2\\
        \text{and } \widehat{\HO}_{n,j}(x) &= (x-x_j)*L_{n,j}(x)^2
    \end{align*}
    Here $L_{n,j}(x)$ are the Lagrange polynomials.
    It's easy to see that $\HO_{n,j}(x_i) = \delta_{ij}$ ($L_{n,j}(x_i) = \delta_{ij}$)
    and $\widehat{\HO}_{n,j}(x_i) = 0$ $\forall x_i$

    \begin{align*}
        \HO_{n,j}'(x) &= -2 L'_{n,j}(x_j)*L_{n,j}(x)^2 + 2[1-2(x-x_j)L_{n,j}'(x_j)]*L_{n,j}'(x)*L_{n,j}(x)\\
        \HO_{n,j}'(x_j) &= -2L_{n,j}'(x_j)+2L_{n,j}'(x_j) = 0\\
        H_{n,j}(x_i) |_{i \neq j} &= 0\\
        \\
        \widehat{\HO}_{n,j}'(x)&=L_{n,j}(x)^2+2(x-x_j)L_{n,j}'(x)L_{n,j}(x)\\
        \widehat{\HO}_{n,j}'(x_j)&= 1\\
        \widehat{\HO}_{n,j}'(x_i) |_{i \neq j} &= 0
    \end{align*}
\end{theorem}

