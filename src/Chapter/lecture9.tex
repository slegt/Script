\begin{equation*}
    \frac{\partial ( \xi ' ) }{\partial a_K} = 0
\end{equation*}
\textrightarrow{} $m+1$ equations for $m+1$ parameters $a_0, \ldots, a_m$.

\begin{align*}
    &\frac{\partial}{\partial a_k} \left(\sum_{i=1}^{n}(y_i-\sum_{l=0}^{m} a_l x_i^l)^2\right)
    = 2 * \sum_{i=1}^{n} \left(\left(y_i - \sum_{l=0}^{m} a_l x_i^l\right)(-x_i^k)\right)
    = 0 \\
\end{align*}
\begin{equation*}
    \sum_{i=1}^{n} \sum_{l=0}^{m} a_l x_i^{k+l} = \sum_{i=1}^{n} y_i x_i^k
\end{equation*}

\begin{align*}
    a_0 \sum_{i=1}^{n} x_i^0 + a_1 \sum_{i=1}^{n} x_i^1 + \ldots + a_m \sum_{i=1}^{n} x_i^m &= \sum_{i=1}^{n} y_i x_i^0\\
    a_0 \sum_{i=1}^{n} x_i^1 + a_1 \sum_{i=1}^{n} x_i^2 + \ldots + a_m \sum_{i=1}^{n} x_i^{m+1} &= \sum_{i=1}^{n} y_i x_i^1\\
    \ldots \\
    a_0 \sum_{i=1}^{n} x_i^m + a_1 \sum_{i=1}^{n} x_i^{m+1} + \ldots + a_m \sum_{i=1}^{n} x_i^{2m} &= \sum_{i=1}^{n} y_i x_i^m\\
\end{align*}

again a matrix $C_{ij} = \sum_{k=1}^{n} x_k^{i+j-2}$ has to be inverted.

Even more complicated: general linear model
\begin{equation*}
    F_m(x) = \sum_{k=1}^{m} a_k f_k(x)
\end{equation*}
$F_m$ is linear in $a_k$, $f_n$ can be nonlinear in $x$.
(Error weighted $\sigma \to \sigma_i$)
\begin{align*}
    \text{residues } r_i &= \frac{1}{\sigma_i}(F_m(x_i) - y_i)\\
    &= \sum_{k=1}^{m} a_k \frac{f_k(x_i)}{\sigma_i}- \frac{y_i}{\sigma_i} \\
    &= \sum_{k=1}^{n} a_k C_{ik} - \tilde{y}_i\\
    \text{with } C_{ik} &= \frac{f_k(x_i)}{\sigma_i} \text{ and } \tilde{y}_i = \frac{y_i}{\sigma_i}
\end{align*}
\begin{align*}
    \underline{r} &= C \underline{a} - \underline{\tilde{y}}\\
    (\underline{r}^2) &= (C \underline{a} - \underline{\tilde{y}})^2
    =(C \underline{a} - \underline{\tilde{y}})^T (C \underline{a} - \underline{\tilde{y}})\\
    &= (\underline{a}^T C^T - \underline{\tilde{y}}^T) (C \underline{a} - \underline{\tilde{y}})
    = \underline{a}^T C^T C \underline{a}  - 2 \underline{\tilde{y}}^T C \underline{a} + \underline{\tilde{y}}^T \underline{\tilde{y}}\\
    \nabla_a (\underline{r}^2)
    &= 2 \underline{a}^T C^T C - 2 \underline{\tilde{y}}^T C
    = 2 ( C \underline{a} - \underline{\tilde{y}})^T C = 0 \text{ (minimum)}\\
    C^T C \underline{a} &= C^T \underline{\tilde{y}}\\
\end{align*}
To get the desired solution we have to invert $C^t C$


\chapter{Numerical differentiation and integration}\label{ch:numerical-differentiation-and-integration}


\section{differentiation}\label{sec:differentiation}
Problem: We can evaluate some function $f(x_i)$ but not its derivative.
\begin{align*}
    f'(x_0) &= \lim_{h \to 0} \frac{f(x_0 + h) - f(x_0)}{h}\\
    (f &\in C^2([a,b]), x_0 \in [a,b])
\end{align*}
First-order interpolating Lagrange polynomial through $(x_0, f(x_0)), (x_1, f(x_1))$
\begin{equation*}
    f(x) = \frac{x-x_1}{x_0-x_1} f(x_0) + \frac{x-x_0}{x_1-x_0} f(x_1) + \frac{(x-x_0)(x-x_1)}{2} f''(\xi(x))
\end{equation*}
set $h= x_1 - x_0$ not too small to keep rounding errors small.
\begin{align*}
    f(x) &= \frac{x-x_0 - h}{-h} f(x_0) + \frac{x-x_0}{h} f(x_0 + h) + \frac{(x-x_0)(x-x_0-h)}{2} f''(\xi(x))\\
    f'(x) &= \frac{f(x_0 + h) - f(x_0)}{h} + \frac{2 (x-x_0) - h}{2} f''(\xi(x))
    + \underbrace{\frac{(x-x_0)(x-x_0-h)}{2} \frac{d}{dx}f''(\xi(x))}_{\text{0 if } x = x_0 \text{ or } x = x_0 + h}\\
    f'(x_0) &= \underbrace{\frac{f(x_0 + h) - f(x_0)}{h}}_{\text{Approximation of } f'(x_0)} - \underbrace{\frac{h}{2} f''(\xi(x))}_{\substack{\text{Error of the} \\ \text{approximation}}}
    \to \text{two-point formula}
\end{align*}

For general formulae with more than 2 points:
let $x_0 < x_1 < \ldots < x_n$ be pairwise distinct points in $[a,b]$ and $f \in C^{n+1}([a,b])$.

\begin{align*}
    f(x) &= \sum_{l=0}^{n} f(x_l) \underbrace{L_l(x)}_{\substack{\text{Lagrange} \\ \text{polynomials}}} + \underbrace{\frac{(x-x_0)(x-x_1)\ldots(x-x_n)}{(n+1)!} f^{(n+1)}(\xi(x))}_{\text{error of interpolation}}\\
    f'(x) &= \sum_{l=0}^{n} f(x_l) L_l'(x) + \frac{d}{dx}\left[\frac{(x-x_0)\ldots(x-x_n)}{(n+1)!}\right] f^{(n+1)}(\xi(x))
    + \frac{(x-x_0)\ldots(x-x_n)}{(n+1)!} \frac{d}{dx} f^{(n+1)}(\xi(x))
\end{align*}
if $x = x_k$ the last term vanishes and
\begin{equation*}
    f'(x_k) = \sum_{l = 0}^{n} f(x_l) L_l'(x_k) + \frac{f^{n+1} (\xi(x_k))}{(n+1)!} \prod_{j = 0, j \neq k}^{n} (x_k - x_j)
\end{equation*}
for $n = 2$, $x_1 = x_0 + h, x_2 = x_0 + 2h$ we get
\begin{align*}
    \sum_{l = 0}^{n} f(x_l) L_l'(x_k) &= \frac{(x-x_0-h)(x-x_0-2h)}{(-h)(-2 h)} f(x_0) + \frac{(x-x_0)(x-x_0-2h)}{h(-h)} f(x+h)\\
    &\phantom{=}+ \frac{(x-x_0)(x-x_0-h)}{2h*h} f(x_0+2h) \text{ (\textasteriskcentered)}\\
    \frac{d}{dx} ( \text{\textasteriskcentered} ) &= \frac{2(x-x_0-h)-h}{2h^2}f(x_0) + \frac{2(x-x_0)-2h}{-h^2} f(x_0+h) + \frac{2(x-x_0)-h}{2h^2} f(x_0+2h)\\
\end{align*}
\begin{align*}
    \sum_{l = 0}^{n} f(x_l) L_l'(x_0) = -\frac{3}{2h} f(x_0) + \frac{2}{h} f(x_0+h) - \frac{1}{2h} f(x_0+2h)\\
    \text{error: } \frac{f'''(\xi(x))}{3!} (x-x_0-h)(x_0-x_0-2h) = \frac{h^2}{3} f''(\xi(x))\\
    f'(x_0) = \frac{1}{2h} (-3 f(x_0) + 4 f(x_0+h) - f(x_0+2h)) + \frac{h^2}{3} f''(\xi(x))\\
    \text{for some } $\xi \in [x_0, x_0+2h]$ \to \text{ three-point endpoint formula}.
\end{align*}
This formula is useful if the derivative at the endpoints of an interval are required.
(Spline with Hermite boundaries, $h > 0 $ left side, $h < 0$ right side)

For derivatives at an inner node we use neighboring nodes on either side.
For 2nd order polynomial $x_0-h, x_0, x_0+h$ one obtains the three-point midpoint formula.

\begin{align*}
    f'(x_0) &= \frac{1}{2h} (f(x_0+h) - f(x_0-h)) - \frac{h^2}{6} f'''(\xi(x))\\
    \xi &\in [x_0-h, x_0+h]\\
\end{align*}
The error is only half as large as for the endpoint formula and one only needs to evaluate $f$ twice.
(``three-point'' because $f(x_0)$ has been taken into account conceptually and has pre-factor zero).
It is also preferable to the two-point formula since the error term it of a higher order in h.

Using a fourth-order polynomial one obtains the five-point midpoint formula:
\begin{equation*}
    f'(x_0) = \frac{1}{12h} (f(x_0-2h) - 8 f(x_0-h) + 8 f(x_0+h) - f(x_0+2h)) + \frac{h^4}{30} f^{(5)}(\xi(x))\\
\end{equation*}
as well as the five-point endpoint formula:
\begin{equation*}
    f'(x_0) = \frac{1}{12h} (-25 f(x_0) + 48 f(x_0+h) - 36 f(x_0+2h) + 16 f(x_0+3h) - 3 f(x_0+4h)) + \frac{h^4}{5} f^{(5)}(\xi(x))\\
\end{equation*}

\begin{example}
    \begin{equation*}
        f(x) = x * e^x
    \end{equation*}
    \begin{center}
        \begin{tabular}{r l}
            \toprule
            $x$ & $f'(x)$   \\
            \midrule
            1.8 & 10.889365 \\
            1.9 & 12.703199 \\
            2.0 & 14.778112 \\
            2.1 & 17.148957 \\
            2.2 & 19.855030 \\
        \end{tabular}
    \end{center}
    \begin{equation*}
        f'(2.0) = 22.167168
    \end{equation*}
    \begin{align*}
        &\text{three-point endpoint h = 0.1} &\frac{1}{0.2}(-3*f(2.0) + 4*f(2.1) - f(2.2)) &= 22.032310\\
        &\text{three-point endpoint h = -0.1} &-\frac{1}{0.2} (-3*f(2.0) + 4*f(1.9) - f(1.8)) &= 22.054525\\
        &\text{three-point midpoint h = 0.1} &\frac{1}{0.2} (f(2.1) - f(1.9)) &= 22.228790\\
        &\text{three-point midpoint h = 0.2} &\frac{1}{0.4} (f(2.2) - f(1.8)) &= 22.41426\\
        &\text{five-point midpoint h = 0.1} &\frac{1}{1.2}(f(1.8)-8*f(1.9)+8*f(2.1)-f(2.2)) &= 22.166999\\
    \end{align*}
\end{example}