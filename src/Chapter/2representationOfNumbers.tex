\chapter{Number Systems}
In the computer, the smallest units of memory is the bit, which can assume
just 2 different states \\
\{0, 1\} \{on, off\}, \{true, false\} , not necessary numbers.
Larger units of memory are:
\begin{center}
    \begin{tabular}{ l l @{ }l}
        \toprule
        name & \multicolumn{2}{l}{value} \\
        \midrule
        Byte     & 8        & bits  \\
        Kibibyte & $2^{10}$ & bytes \\
        kilobyte & $10^3$   & bytes \\
        Mebibyte & $2^{20}$ & bytes \\
        Megabyte & $10^6$   & bytes \\
        \bottomrule
    \end{tabular}
\end{center}


\section{Integer numbers}\label{sec:integer-numbers}
Integer numbers can easily be represented.
The data ``standards`` are called ``data types`` and specify the size and value function.
E.g.: 1 byte unsigned integer;
\begin{center}
    \begin{tabular}{ l | l }
        memory state & value  \\
        \hline
        11111111     & 255    \\
        \ldots       & \ldots \\
        00000001     & 1      \\
        00000000     & 0
    \end{tabular}
\end{center}

with negative numbers \textrightarrow{} 1 byte signed integer:
\begin{center}
    \begin{tabular}{ l | l }
        momory state & value  \\
        \hline
        01111111     & 127    \\
        \ldots       & \ldots \\
        00000001     & 1      \\
        00000000     & 0      \\
        11111111     & -1     \\
        11111110     & -2     \\
        11111101     & -3     \\
        \ldots       & \ldots \\
        10000000     & -128
    \end{tabular}
\end{center}


\section{Floating point numbers (fractions)}\label{sec:floating-point-numbers-(fractions)}
Floating-point numbers are the product of number $m \in (0,1)$ called mantissa, an integer e power of
a base b (usually b = 2) and a sign s.
\begin{equation*}
    x = (-1)^s * m * b^e
\end{equation*}
{}
Datatype ``single precision`` / 32-bit float
\begin{itemize}
    \item sign $(-1)^s$
    \item exponent e is an integer, but it is stored differently than ``normal`` integers - binary number is shifted by 127

    \begin{tabular}{l | l | l}
        memory state & binary number & value of e \\
        \hline
        11111111     & 255           & 128        \\
        11111110     & 254           & 127        \\
        \ldots       & \ldots        & \ldots     \\
        100000000    & 128           & 1          \\
        01111111     & 127           & 0          \\
        \ldots       & \ldots        & \ldots     \\
        00000010     & 2             & -125       \\
        00000001     & 1             & -126       \\
        00000000     & 0             & -127
    \end{tabular}

    The values $e = 127$ and $e = 128$ are not used in the usual way.
    Instead, they are reserved for special cases (e.g. $\pm 0$, $ \pm \infty$, signaling NaN)
    \item mantissa is a positive binary number.
    In order to achieve maximal precision e is chosen such that all positions of m are used (no leading zeros).
    The point is after the first digit.

    E.g.: 4-digit decimal mantissa:
    \begin{align*}
        310678 &\rightarrow 3.107 * 10^5 \\
        0.00043136 &\rightarrow 4.314 * 10^{-4}
    \end{align*}

    Hence, if $x \neq 0$ there is always a non-zero digit in front of the point.
    Since $b = 2$ this digit is a 1.
    It is therefore not stored, but set implicitly.
    (This is only not the case if $e=-127$)

    \begin{align*}
        m = 1 + \sum_{i = 1}^{23} m_i * 2^{-i}
    \end{align*}
\end{itemize}

The datatype double precision (float64, double)
*image*
\begin{align*}
    e \in \{ -1023, \ldots, 1023 \} \\
    (-1023, 1024 \hbox{ reserved})
\end{align*}

\emph{Precision:} The relative precision with which a number
can be represented is determined by the number of bits of the mantissa $m \in [1, 2 )$

23 bits \textrightarrow{} $2^{23} \approx 10^7$ different states of m \textrightarrow{} relative distance of a possible number x to its ``neighbor'' is $10^{-7}$


double precision: 52 bits \textrightarrow $2^{-52} \approx 10^{-15}$
