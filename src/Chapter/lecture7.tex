In summary we obtain
\begin{align*}
    \HO_{2n+1}(x) = \sum_{j=0}^{n} f(x_j)*\HO_{n,j}(x)+ \sum_{j=0}^{n} f'(x_j) \widehat{\HO}_{n,j}(x)
\end{align*}
\begin{center}
    \begin{tabular}{c c}
        \[ \HO_{n,j}(x_i) = \delta_{ij}\] & \[\HO_{n,j}'(x_i) = 0\]                      \\
        \[ \widehat{\HO}_{n,j}(x_i) = 0\] & \[ \widehat{\HO}_{n,j}'(x_i) = \delta_{ij}\]
    \end{tabular}
\end{center}
However, the construction by means of Lagrange polynomials is computationally expensive.
Again, divided differences are useful.
From the remainder of polynomial interpolation it follows:
\begin{lemma}
    Let $f \in C^n([a,b])$ and $x_0, \ldots, x_n$ be pairwise distinct in $[a,b]$.
    Then there is a $\xi \in [a,b]$ such that
    \begin{equation*}
        f[x_0,\ldots, x_n] = \frac{f^{(n)}(x)}{n!} \text{ (\textasteriskcentered)}
    \end{equation*}
\end{lemma}
\begin{proof}
    If $n+1$ nodes $x_0, \ldots, x_n$ and the values $f(x_i), f'(x_i)$ $(i=0, \ldots, n)$ are given, we define a sequence
    \begin{equation*}
        \hat{x}_{2i} = \hat{x}_{2i+1} = x_i
    \end{equation*}
    We generate divided differences with sequence.
    Obviously
    \begin{equation*}
        f[\hat{x_}{2i}, \hat{x}_{2i+1}]
        = \frac{f[\hat{x}_{2i+1}]-f[\hat{x}_{2i}]}{\hat{x}_{2i+1}-\hat{x}_{2i}}
    \end{equation*}
    can not be used.
    But considering $(\textasteriskcentered)$ and
    \begin{equation*}
        \lim_{x_A \to x_0} f[x_0, x_A] = f'(x_0)
    \end{equation*}
    it is justified to use the substitution
    \begin{equation*}
        f[\hat{x}_{2i}, \hat{x}_{2i+1}] = f'(x_i)
    \end{equation*}
    and use $f'(x_0), f'(x_1), f'(x_2), \ldots, f'(x_n)$ for
    $f[\widehat{x_0}, \widehat{x_1}], f[\widehat{x_2}, \widehat{x_3}] f[\widehat{x_4}, \widehat{x_5}],
    \ldots, f[\widehat{2n}, \widehat{x_{2n+1}}]$
    The remaining divided differences are generated as before.
\end{proof}

\section{Spline interpolation}\label{sec:spline-interpolation}
Although an interpolating polynomial can always be found, this is often not a very good approximation, especially in the case of large n.
Fluctuation in a small region can lead to great changes throughout the entire interval.
Piecewise interpolation can be preferable.

The simplest version of piecewise interpolation would employ polynomials of degree 1.
This has the big disadvantage that the resulting curve is not smooth/differentiable.
Better choice: Piecewise hermite interpolation with polynomials of degree 3.
A cubic spline interpolating function is a function $s$ with the properties
\begin{itemize}
    \item $s$ is a cubic polynomial labeled $s_j$ on the subinterval $[x_j, x_{j+1}]$
    \item $s_j(x_j) = f(x_j)$ for $j = 0, \ldots, n$
    \item $s_{j+1}(x_{j+1}) =s_j(x_{j+1})$
    \item $s_{j+1}'(x_{j+1}) =s_j'(x_{j+1})$
    \item $s_{j+1}''(x_{j+1}) =s_j''(x_{j+1})$
    \item one of the following boundary conditions
    \begin{itemize}
        \item $s''(x_0)=s''(x_n) = 0$ (free boundary)
        \item $s'(x_0) = f'(x_0)$ and $s'(x_n) = f'(x_n)$ (hermite boundary)
    \end{itemize}
\end{itemize}


\chapter{Trigonometric interpolation}\label{ch:trigonometric-interpolation}
We consider the n-dimensional space $T_c^{n-1}$ of complex trigonometric polynomials
\begin{align*}
    \Phi_n(x) = \sum_{k=0}^{N-1} c_k * e^{ikx} \text{ with } c_k \in \mathbb{C}
\end{align*}
of degree $N-1$.
These polynomials are periodic: $\Phi_n(x)=\Phi_n(2 \pi + x)$.
The linear independence of the basis functions $e^{ikx}$ guarantees that for $N$ data-points $(x_j, f_j)$ with $j= 0, \ldots, N-1$ there is exactly
one interpolating polynomial $\Phi_n$ with $\Phi_n(x_j)=f_j$.
If we constrain ourselves to equidistant nodes $x_j = \frac{2  \pi j}{N}$ then we obtain the Vandermond-System.
\begin{equation*}
    \begin{pmatrix}
        1      & w_0     & w_0^2     & \dots  & w_0^n         \\
        \vdots & \vdots  & \vdots    & \ddots & \vdots        \\
        1      & w_{n-1} & w_{n-1}^2 & \dots  & w_{n-1}^{n-1}
    \end{pmatrix}
    *
    \begin{pmatrix*}
        c_0\\
        \vdots \\
        c_{n-1}
    \end{pmatrix*}
    =
    \begin{pmatrix*}
        f_0\\
        \vdots\\
        f_{n-1}
    \end{pmatrix*}
\end{equation*}
where $w_k$ are the roots of unity of degree $N$.
\begin{align*}
    w_k \coloneq e^{ikx} &= e^{i*2 \pi k / N}\\
    (w_k)^N &= 1
\end{align*}
\begin{lemma}
    The basis functions $e^{ikx}$ are orthogonal with respect to the inner product
    \begin{equation*}
        \langle f, g \rangle \coloneq \frac{1}{n} * \sum_{j=0}^{N-1} f(x_j) * \overline{g(x_j)}
    \end{equation*}
    with the nodes $x_j = {2\pi j}/{N}$
\end{lemma}
\begin{proof}
    \begin{align*}
    (w_k)
        ^l &= (w_l)^k = w_1^{k*l}\\
        \left( e^{i*2 \pi k / N} \right)^l &= \left( e^{i*2 \pi l / N} \right)^k = e^{i*2 \pi kl / N}\\
        \\
        \langle e^{ikx}, e^{ilx} \rangle &= \frac{1}{N} * \sum_{j=0}^{N-1} e^{i*2 \pi k j / N} * e^{-i*2 \pi l j / N}\\
        &= \frac{1}{n} \sum_{j=0}^{N-1} \underbrace{w_j^m w_j^{-l}}_{w_j^{k-l}} = N * d_{kl} \text{ is equivalent to}\\
        \sum_{j=0}^{N-1}w_j^k &= \sum_{j=0}^{N-1} w_k^{j} = N * \delta_{k0}
    \end{align*}
    Roots of unity $w_k$ are solutions of $0 = w^n-1 = (w-1)(w^{n-1}+w^{n-2}+\ldots+1) = (w-1) \sum_{j=0}^{N-1} w_k^j$
    If $k \neq 0$ then $w_k \neq 1$ and therefore $\sum_{j=0}^{N-1} w_k^j = 0$.
    If $k=0$ then $\sum_{j=0}^{N-1} w_k^j = N$.
    The coefficients $c_n$ of the trigonometric interpolation of the $N$ data points $(x_l, f_l)$ with equidistant nodes $x_l = \frac{2 \pi l}{N}$ are given by
    \begin{align*}
        c_k &= \sum_{l=0}^{N-1} f_l * w_l^{-k} \text{ for } k = 0, \ldots, N-1\\
        &= \langle f, e^{ikx} \rangle
    \end{align*}

    Proof by insertion:
    \begin{align*}
        \Phi(x_m)=\sum_{k=0}^{N-1} c_k *w_m^k &= \sum_{k=0}^{N-1} \frac{1}{N} \sum_{l=0}^{N-1} f_l * w_l^{-k} w_m^k\\
        &= \frac{1}{N} \sum_{l=0}^{N-1} f_l \sum_{k=0}^{N-1} \underbrace{w_l^{-k}*w_m^k}_{w_k^{m-l}}\\
        &= \frac{1}{N} \sum_{l=0}^{N-1} f_l * N * \delta_{ml} = f_m
    \end{align*}
\end{proof}
\begin{remark}
    If the polynomial is real
    \begin{equation*}
        \phi_N(x) = \sum_{k=0}^{N-1}c_{k}*e^{ikx} \in \mathbb{R} \text{ for all } x \in \mathbb{R}
    \end{equation*}
    one can use the representation
    \begin{align*}
        \text{for odd N: }\Phi_{2n+1}(x) &= \frac{a_0}{2}+\sum_{k=1}^{n} a_k * \cos(kx) + b_k * \sin(kx) \text{ with } n = \frac{N-1}{2}\\
        \text{for even N: } \Phi_{2n}(x) &= \frac{a_0}{2}+\sum_{k=1}^{n-1} a_k * \cos(kx) + b_k * \sin(kx) + \frac{a_n}{2} * \cos(nx) \text{ with } n = \frac{N}{2}
    \end{align*}
    with $a_k, b_k \in \mathbb{R}$.
    It is $a_k = c_k + c_{N-k}$ and $b_k = i *(c_k-c_{N-k})$
\end{remark}
\begin{remark}
    With $c_{-j} = c_{N-j}$ we can write for odd N
    \begin{align*}
        \Phi_n(x_l) = \sum_{k=0}^{N-1} c_k e^{ikx_l} = \sum_{k = -n}^{n} c_k *e^{ikx_l}  
        \text{ with } n = \frac{N-1}{2}\\
        e^{-ikx_l} = e^{-ik \frac{2 \pi l}{n}} = e^{-i k \frac{2 \pi l}{N}} * \underbrace{e^{i 2 \pi l}}_{=1^l}
        = e^{i (N-k) \frac{2 \pi l}{N}}
    \end{align*}
    Strong similarity to truncated Fourier series:
    \begin{equation*}
        f_n(x) = \sum_{k = -n}^{n} \widehat{f_k} * e^{ikx}
    \end{equation*}
    with coefficients
    \begin{equation*}
        \widehat{f_k} = \frac{1}{2 \pi } \integral{0}{2 \pi}{f(x) * e^{-ikx}}
    \end{equation*}
    If the integral is approximated as a sum of the x values $x_l = \frac{2 \pi l}{N}$ one
    obtains an approximation of $\widehat{f_k}$
    \begin{align*}
        \integral{0}{2 \pi}{g(x)} \approx \frac{2 \pi }{N} \sum_{k = 0}^{N-1}g(x_k)\\
        \widehat{f_k} \approx \frac{1}{N} \sum_{l = 0}^{N-1} f_l e^{-ikx_l} = \frac{1}{N}
        \sum_{l=0}^{N-1} e^{-ikl \frac{2 \pi}{N}} f_l = c_l
    \end{align*}
    Therefore the isomorphism
    \begin{equation*}
        \widetilde{f_n}: \mathbb{C} \to \mathbb{C}: (f_j) \to (\c_j)
    \end{equation*}
    is also called discrete Fourier-Transformation.
    \begin{align*}
        c_k = \frac{1}{N} \sum_{l=0}^{N-1} f_l e^{-i 2 \pi k l / N}\\
        f_k = \sum_{l=0}^{N-1} c_l e^{i 2 \pi k l / N}
    \end{align*}
\end{remark}

